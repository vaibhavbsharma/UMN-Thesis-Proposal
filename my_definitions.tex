% Physics constants
\newcommand{\C}{{\mathrm{c}}}

% Add space between rows of tables
\newcommand{\spacerows}[1]{\renewcommand{\arraystretch}{#1}}

% Define a better looking eV by moving the V slightly left
\DeclareSIUnit\electronvolt{e\hspace{-0.08em}V}

\newcommand*{\permcomb}[4][0mu]{{{}^{#3}\mkern#1#2_{#4}}}
\newcommand*{\perm}[1][-3mu]{\permcomb[#1]{P}}
\newcommand*{\comb}[1][-1mu]{\permcomb[#1]{C}}

\newcommand{\widthfactor}{0.7}
%    \usepackage{lineno}
% http://tex.stackexchange.com/questions/7996/lineno-and-syntax-package-incompatibilities
%    \makeatletter
%    \def\gr@implitem<#1> #2 {%
%     \sbox\z@{\hskip\labelsep\grammarlabel{#1}{#2}}%
%      \strut\@@@par% lineno.sty redefines \@@par which was in the original code
%     \vspace{-\parskip}%
%      \vspace{-\baselineskip}%
%      \hrule\@height\z@\@depth\z@\relax%
%      \item[\unhbox\z@]%
%      \catcode`\<\active%
%    }
%    \makeatother

\lstset{language=C}

\lstdefinestyle{nonumbers}
{numbers=none}

\definecolor{mygreen}{rgb}{0,0.4,0}
\definecolor{mygray}{rgb}{0.5,0.5,0.5}
\definecolor{mymauve}{rgb}{0.58,0,0.82}
\lstset{ %
backgroundcolor=\color{white},   % choose the background color; you
%must add \usepackage{color} or \usepackage{xcolor}
basicstyle=\ttfamily\small,        % the size of the fonts that are used
%for the code
basewidth = {.5em, 0.5em},
%breakatwhitespace=false,         % sets if automatic breaks should
%only happen at whitespace
breaklines=true,                 % sets automatic line breaking
captionpos=b,                    % sets the caption-position to bottom
commentstyle=\color{mygreen},    % comment style
deletekeywords={...},            % if you want to delete keywords from
%the given language
escapeinside={\%*}{*)},          % if you want to add LaTeX within
%your code
extendedchars=true,              % lets you use non-ASCII characters;
%for 8-bits encodings only, does not work with UTF-8
frame=single,	                   % adds a frame around the code
keepspaces=true,                 % keeps spaces in text, useful for
%keeping indentation of code (possibly needs columns=flexible)
keywordstyle=\color{blue},       % keyword style
language=C,                 % the language of the code
otherkeywords={reg8\_t,reg64\_t},           % if you want to add more keywords to
%the set
numbers=left,                    % where to put the line-numbers;
%possible values are (none, left, right)
numbersep=5pt,                   % how far the line-numbers are from
%the code
numberstyle=\tiny\color{black}, % the style that is used for the
%line-numbers
rulecolor=\color{black},         % if not set, the frame-color may be
%changed on line-breaks within not-black text (e.g. comments (green
%here))
showspaces=false,                % show spaces everywhere adding
%particular underscores; it overrides 'showstringspaces'
%  showstringspaces=false,          % underline spaces within strings
%only
showtabs=false,                  % show tabs within strings adding
%particular underscores
stepnumber=1,                    % the step between two line-numbers.
%If it's 1, each line will be numbered
stringstyle=\color{mymauve},     % string literal style
tabsize=2,	                   % sets default tabsize to 2 spaces
%  title=\lstname                   % show the filename of files included
%with \lstinputlisting; also try caption instead of title
literate={->}{$\rightarrow$}{2}
{α}{$\alpha$}{1}
{δ}{$\delta$}{1}
}

% declare the floating environment {Grammar}
% this will also define \listofGrammars:
\DeclareFloatingEnvironment[
% the file extension for the file used to create the list:
fileext   = logr,% don't use log here!
% the heading for the list:
listname  = {List of Grammars},
% the name used in captions:
name      = Grammar,
% the default floating parameters if the environment is used
% without optional argument:
placement = htp
]{Grammar}