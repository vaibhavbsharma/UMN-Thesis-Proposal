%%%%%%%%%%%%%%%%%%%%%%%%%%%%%%%%%%%%%%%%%%%%%%%%%%%%%%%%%%%%%%%%%%%%%%%%%%%%%%%%
% abstract.tex: Abstract
%%%%%%%%%%%%%%%%%%%%%%%%%%%%%%%%%%%%%%%%%%%%%%%%%%%%%%%%%%%%%%%%%%%%%%%%%%%%%%%%
% Abstract
%%%%%%%%%%%%%%%%%%%%%%%%%%%%%%%%%%%%%%%%%%%%%%%%%%%%%%%%%%%%%%%%%%%%%%%%%%%%%%%%
Independently developed codebases typically contain many segments of
code that perform same or closely related operations (semantic
clones). Finding functionally equivalent segments enables applications
like replacing a segment by a more efficient or more secure
alternative. Such related segments often have different interfaces, so
some glue code (an adapter) is needed to replace one with the other.
We present an algorithm that searches for replaceable code segments by
attempting to synthesize an adapter between them from some finite family of
adapters; it terminates if it finds no possible adapter. In the first part of this report, we compare binary symbolic
execution-based adapter search with concrete adapter enumeration based on Intel's
Pin framework, and explore the relation
between size of adapter search space and total search time. We present
examples of applying adapter synthesis for improving security of
binary functions and switching between binary implementations of RC4.
We present two large-scale evaluations: (1) we run adapter synthesis on more
than 13,000 function pairs from the Linux C library, and (2) we
reverse engineer fragments of ARM binary code by running more than a
million adapter synthesis tasks. Our results confirm that several
instances of adaptably equivalent binary functions exist in real-world
code, and suggest that adapter synthesis can be applied for automatically replacing binary code with its
adaptably equivalent variants.

The use of symbolic execution during adapter search implicitly relies on being able to summarize the entire
adapter into a single formula.
%
Summarizing multiple execution paths into a single path is done using path-merging techniques in symbolic execution.
%
Merging related execution paths is a powerful technique for reducing
path explosion in symbolic execution.
%
One approach, introduced and dubbed ``veritesting'' by Avgerinos et
al., works by statically translating a bounded control flow region
into a single formula.
%
This approach is a convenient way to achieve path merging as a
modification to a pre-existing single-path symbolic execution engine.
%
Avgerinos et al. evaluated their approach in a symbolic execution tool
for binary code, but different design considerations apply when
building tools for other languages.
%
In the second part of this report, we explore the best way to use a veritesting approach in
the symbolic execution of Java.
%
Because Java code typically contains many small dynamically dispatched
methods, it is important to include them in multi-path regions; we
introduce a {\em higher-order} path-merging technique to do so
modularly.
%
Java's typed memory structure is very different from a binary, but we
show how the idea of static single assignment (SSA) form can be
applied to object references to statically account for aliasing.
%
We extend path merging to summarize multiple exit points that return control flow from a multi-path region
into a single such exit point.
%More formally, we describe our veritesting algorithms as
%syntax-directed transformations of a structured intermediate
%representation, which highlights their logical structure.
%
We have implemented our algorithms in Java Ranger, an extension to the
widely used Symbolic Pathfinder tool for Java bytecode.
%
Our empirical evaluation shows that Java Ranger greatly reduces the
search space of Java symbolic execution benchmarks with its expanded
path-merging capabilities providing a significant improvement.

Bugs in commercial software and third-party components are an undesirable and expensive phenomenon.
%
Such software is usually released to users only in binary form.
%
The lack of source code renders users of such software dependent on their software vendors for repairs of bugs.
%
Such dependence is even more harmful if the bugs introduce new vulnerabilities in the software.
%
Being able to automatically repair security and functionality bugs breaks this dependence and increases software robustness.
%
In the third part of this report, I will propose development of a binary program repair tool that uses
existing bug-free fragments of code to repair buggy code.