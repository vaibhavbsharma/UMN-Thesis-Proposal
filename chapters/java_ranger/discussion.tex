\section{Future Work}
\label{sec:futureWork}
While Java Ranger has the potential to scale to large real-world Java programs, there are a number of directions along
which it can be further improved.
\begin{itemize}

\item Java Ranger attempts to perform path merging as aggresively as possible. This path merging strategy doesn't
optimize towards making fewer solver calls. We plan to work towards implementing heuristics that
can measure the effect of path merging on the rest of the program.
    
%\item In some models, it is possible to reduce the number of paths to one.  The static region approach essentially
%constructs a unrolled version of the program, similar to what tools like CBMC construct.   This can only happen on
%relatively static models that do not have a lot of object construction leading to multiple dispatch paths.  HOSRs are
%more flexible for these situations and allow specialization depending on dispatch type, which we believe will lead to
%better performance for highly-dynamic models.
%
%\item In general, the solver time does not rise dramatically for disjunctive paths. Since (in the limit) we reduce the
%number of paths exponentially by removing branches, we can perform relatively expensive analyses as preprocessing steps
%and at instantiation if we are able to instantiate a static region, and still end up with much better performance.

\item Java Ranger is most useful when it can merge multiple execution paths into one summary. But such merging causes an
increase in the size of the summary, and consequently, the path condition.
%
Once a summary has been communicated to the solver, it need not be sent again as long as it remains the same.
%
This requires us to use the solver in an incremental mode, where we use previously-constructed state in the solver for
future solving.
%
Our current implementation of Java Ranger sends the entire path condition to the solver with every query, making every
query even more expensive in the presence of large multi-path region summaries.
%
We plan to integrate incremental solving with Java Ranger in the future.
%
\item Multiple return instructions that appear on all execution paths inside a multi-path region can be simplified to a
single return value of the region.
%
We implemented this transformation manually in TCAS but we plan to automate it in the future.
%
Such automation would allow us to summarize regions with return instructions.
%
\item While statically summarizing regions gives dynamic symbolic execution a
performance boost to explore more paths efficiently, generating test cases that covers all summarized branches is one of
the fundamental roles of dynamic symbolic execution that is currently unsupported.
%
This is an extension that we intend to investigate in our future work.
%
\item While path merging can potentially allow symbolic execution to explore interesting parts of a program sooner, the
effect of path merging on search strategies, such as depth-first search and breadth-first search commonly used with
symbolic execution, remains to be investigated. We plan to explore the integration of such guidance heuristics with path merging in the future.
%
\item Java Ranger can summarize methods and regions in Java standard libraries. This creates potential for automatically
constructing summaries of standard libraries so that Java symbolic execution engines can prevent path explosion
originating from standard libraries.
%
\end{itemize}