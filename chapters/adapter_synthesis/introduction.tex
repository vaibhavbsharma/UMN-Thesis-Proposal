\section{Introduction}
%
Given the large corpus of software available today to an average programmer
to reuse, it is desirable to reuse well-tested, bug-free chunks of
code that implement some required functionality.
%
Finding such code can be difficult to automate, and there is no
guarantee that the code found by such a search will have the exact
interface the programmer intends to use. 
%
At such times, programmers find themselves writing wrapper code and
creating unit tests to check if the wrapped code works as they intended.
%
In this paper, we improve upon a previously presented technique~\cite{icst-adaptersynth} that automates the process of
finding functions that match the behavior specified by an existing
function, while also synthesizing the necessary wrapper needed to handle 
interface differences between the original and discovered functions.
%
Use cases for our improved technique include replacing insecure dependencies of off-the-shelf libraries with bug-free variants, reverse engineering binary-level functions by comparing their behavior to known implementations, and locating multiple versions of a function to be run in parallel to provide security through diversity~\cite{BorckBDHHJSS2016}.

Our technique works by searching for a wrapper that can be added around one function's interface to make it equivalent to another function.
%
We consider wrappers that transform function arguments and return values.
%
For example, Listing~\ref{lst:isalpha} shows implementations of the \textit{isalpha} predicate, which checks if a character is a letter, in two commonly-used libraries.
%
Both implementations follow the ISO C standard specification of the \textit{isalpha} function, but the glibc implementation signifies the input is a letter by returning 1024, while the musl implementation returns 1 in that case.
%
The glibc implementation can be adapted to make it equivalent to the musl implementation by replacing its return value, if non-zero, by 1 as shown by the \textit{adapted\_isalpha} function.
%
This illustrates the driving idea of our approach: to check whether two functions, $f_1$ and $f_2$, have different interfaces to the same functionality, we can search for a wrapper that allows $f_1$ to be substituted by $f_2$.

We refer to the function being wrapped around as the {\em reference} function and the function being emulated as the {\em target} function.
%
In the example above, the reference function is \textit{glibc\_isalpha} and the target function is \textit{musl\_isalpha}.
%
We refer to the wrapper code automatically synthesized by our tool as an {\em adapter}.
%
Our adapter synthesis tool searches in the space of all possible adapters allowed by a specified adapter family for an adapter that makes the behavior of the reference function $f_2$ equivalent to that of the target function $f_1$.
%
We represent that such an adapter exists by the notation $f_1 \leftarrow f_2$.
%
Note that this adaptability relationship is not symmetric: $f_1 \leftarrow f_2$ does not imply $f_2 \leftarrow f_1$. 
%
%In the \textit{isalpha} example, the other direction of adapter could be implemented by multiplying the musl implementation\rq s return value with 1024, but in general only one direction of adaptation may be possible.
%
To efficiently search for an adapter, we use counterexample guided inductive synthesis~(CEGIS)~\cite{Solar-LezamaTBSS2006}.
%
An adapter family is represented as a formula for transforming values controlled by parameters: each setting of these parameters represents a possible adapter.
%
Each step of CEGIS allows us to conclude that either a counterexample exists for the previously hypothesized adapter, or that an adapter exists for all previously found tests.
%
We use binary symbolic execution both to generate counterexamples and to find new candidate adapters; the symbolic execution engine internally uses a satisfiability modulo theories~(SMT) solver.
%
We also implement adapter search using a randomly-ordered enumeration of all possible adapters.
%
We always restrict our search to a specified finite family of adapters.
\lstinputlisting[caption={musl and glibc implementations of the \textit{isalpha} predicate and a wrapper around the glibc implementation that makes it equivalent to the musl implementation},
label={lst:isalpha}, style=nonumbers]{chapters/adapter_synthesis/code_samples/musl_glibc.c}
%
We implement adapter synthesis to adapt around inputs and outputs in registers and memory.
%
We demonstrate the use of adapter synthesis for the following applications.
%
\begin{itemize}
\item We demonstrate the use of adapter synthesis for adaptably substituting
a buggy function with its bug-free variant by finding 
adaptable equivalence modulo a bug.
%
\item We demonstrate substitution of RC4 key structure
setup and encryption functions in mbedTLS~(formerly PolarSSL) and
OpenSSL by means of synthesizing adapters between their interfaces. 
%
\item We present an intra-library evaluation of adapter synthesis by
evaluating two of our adapter families on more than 13,000 pairs of
functions from the GNU C library.
%
\item We demonstrate the use of adapter synthesis for reverse engineering at
scale using binary symbolic execution-based adapter synthesis.
%
We show that code fragments in a ARM-based 3rd party firmware image for
the iPod Nano 2g device can be adaptably substituted by reference
functions written for the VLC media player~\cite{vlc}.
%
In this evaluation, we complete more than 1.17 million synthesis tasks,
with each synthesis task navigating an adapter search space of more than
1.353 x $10^{127}$ adapters.
%
%Enumerating this search space concretely would take an infeasible amount of time~($10^{14}$ years).
%
We find our adapter synthesis implementation finds several instances of reference functions in the firmware image.
\end{itemize}

The rest of this chapter is organized as follows.
%
Section~\ref{sec:adapter_synthesis} presents our algorithm for adapter synthesis and describes our adapter families.
%
Section~\ref{sec:implementation} describes our implementation, and
Section~\ref{sec:evaluation} presents examples of application of adapter synthesis, large-scale evaluations, and a comparison of two adapter search implementations.
%
Section~\ref{sec:discussion} discusses limitations and future work,
Section~\ref{sec:related-work} describes related work, and
Section~\ref{sec:conclusion} concludes.
% 
%  Previous introduction written by Stephen  
%
%    It is commonly observed that of the large amount of software being
%    written, there are many instances of repeated functionality.
%    %
%    Such repetition can occur both within one program and between separate
%    programs, and while some duplication occurs by syntactic copying
%    (``cut and paste''), in other cases functionally equivalent code is
%    developed independently by different developers.
%    %
%    Duplication can sometimes be a burden (because it bloats code and
%    allows bugs fixed in one version to persist in another), but it can
%    also be a resource that makes other useful tasks possible.
%    %
%    For instance diverse implementations of the same functionality can be
%    used for fault tolerance, or if implementations differ in performance
%    we can optimize a program by replacing a slower function with a faster
%    one.
%    
%    The most familiar techniques for finding code duplication examine
%    program source code, and search for code segments that are the same at
%    the level of code syntax (syntactic clones).
%    %
%    But for many applications we wish to find functions that have the same
%    behavior even if their implementations are syntactically different
%    (semantic clones).
%    %
%    For security applications, or when operating on legacy software, it
%    can also be desirable to find clones even when some or all of the
%    software involved is available only it its final executable
%    (``binary'') form.
%    %
%    A semantic rather than a syntactic approach is also natural for
%    binaries because the compilation process can introduce many kinds
%    instruction-level variation.
%    %
%    With these applications in mind we explore techniques for finding
%    semantically-equivalent code at the binary level.
%    
%    The application that originally motivated us is using semantically
%    equivalent code segments to make variants of a program with
%    implementation diversity that can help it to withstand faults or even
%    malicious attack.
%    %
%    This can be particularly effective if multiple diverse variants are
%    executed together and checked for
%    divergence~\cite{Cox:2006:NSS:1267336.1267344}.
%    %
%    Such a defense is stronger if the variants are more semantically
%    diverse, but most techniques that automatically produce variant
%    executables from a single source program affect only low-level
%    features like memory layout~\cite{Larsen:2014:SAS:2650286.2650803},
%    and so provide protection only against low-level attacks.
%    %
%    Finding usable alternative implementations in existing code bases is a
%    promising direction for discovering semantic diversity that can
%    provide more robust protection~\cite{BorckBDHHJSS2016}.
%    
%    A challenge for finding semantic equivalence is that code we would
%    intuitively consider to implement the same functionality might have a
%    different interface, especially if it was developed independently.
%    %
%    (For now we concentrate on code segments that implement functions, so
%    the interfaces we consider are function interfaces.)
%    %
%    An equivalence checking technique that was only able to find
%    implementations with exactly the same interface would miss many
%    semantic clones.
%    %
%    Programmers typically bridge incompatible function interfaces by
%    writing wrapper code that we call an {\em adapter}: a function that
%    implements one interface in terms of another one by making small
%    changes such as copying or reformatting data, or rearranging function
%    arguments.
%    %
%    When we can write an adapter that wraps a function $f_2$ to make it
%    equivalent to a function $f_1$, we write $f_1 \leftarrow f_2$.
%    %
%    The function $f_2$ which is inside the wrapper function we call the
%    {\em inner function}, while the function $f_1$ whose functionality we
%    want to emulate we refer to as the {\em target function}.
%    %
%    The combination of the adapter and the inner function, which is
%    semantically equivalent to the target function, we call the {\em outer
%      function}.
%    
%    This suggests that we would like to consider two functions to be
%    semantically equivalent whenever such an adapter function might be
%    written to replace one with the other.
%    %
%    Because such adapters have a stylized form and the specification in
%    terms of function equivalence is precise, we observe that we can turn
%    this intuition into an automated approach by applying program
%    synthesis.
%    %
%    Given a pair of functions, our system searches over a space of
%    possible adapter functions until it either finds an adapter that makes
%    the outer function equivalent to the target function, or proves that
%    no such adapter exists.
%    %
%    Though this basic approach could be applied equally well to source
%    code or binary code, we implement it using binary-level symbolic
%    execution to increase its applicability, including to systems written
%    using a mixture of C and assembly language.
%    
%    It would be challenging to find a correct adapter in a single search,
%    since the spaces of function inputs and possible adapters are in
%    general both large, and an adapter must work for all inputs.
%    %
%    (From a logical perspective, this corresponds to the fact that the
%    full problem has a quantifier alternation: we seek to show that {\em
%      there exists} an adapter which works {\em for all} inputs.)
%    %
%    To make the search more tractable, we use the technique of
%    counterexample-guided inductive synthesis (CEGIS) to split it into an
%    alternation of two kinds of search.
%    %
%    Rather than demanding an adapter that works over all inputs from the
%    start, our algorithm grows a finite set of test cases, searching at
%    each step for an adapter that works for the current set of test cases,
%    and then checking whether that particular adapter works for all
%    inputs.
%    %
%    Both of these simpler search steps can be implemented with symbolic
%    execution and quantifier-free SMT (satisfiability modulo theories)
%    solving.
%    
%    A key factor in making this approach practical is choosing a space of
%    possible adapters which is large enough to find many related
%    functions, but limited enough that synthesis is tractable.
%    %
%    So we investigate a number of different adapter grammars and
%    illustrate them with examples.
%    %
%    We have implemented out approach on top of an existing binary symbolic
%    execution system, and we evaluate it with a large-scale case study
%    looking for semantically-adaptable function pairs in the standard C
%    library used on Linux (eglibc 2.19 from Ubuntu 14.04).
%    
%    The rest of this paper is organized as follows.
%    %
%    Section~\ref{sec:overview} describes more specifically the kind of
%    adapters we search for, with examples.
%    %
%    Section~\ref{sec:adapter_synthesis} gives our algorithm for adapter
%    synthesis and describes our adapter grammar.
%    %
%    Section~\ref{sec:implementation} describes our implementation, and
%    Section~\ref{sec:evaluation} presents the C library case study.
%    %
%    Section~\ref{sec:discussion} discusses limitations and future work,
%    Section~\ref{sec:related-work} describes related work, and
%    Section~\ref{sec:conclusion} concludes.
